\documentclass[12pt]{article}

\usepackage[margin=1.1in]{geometry}
\input{../../syllabi/preamble}

\newcommand{\coursedept}{Math}
\newcommand{\coursenumber}{341}
\newcommand{\coursenumbercrosslisted}{/ 650.03~}
\newcommand{\semester}{Spring}
\newcommand{\numcredits}{3}
\newcommand{\lectimeandloc}{Mon and Wed 3:10 -- 4:25PM / on zoom}
\newcommand{\requiredlabtimeandloc}{}
\newcommand{\tataofficehourtimeandloc}{TA 	/ TA Office Hours / Loc 			& Abhinav Patil / TBD week-week / on zoom}
\newcommand{\coursewebpageurl}{https://github.com/kapelner/QC_\coursedept_\coursenumber_\semester_\the\year}
\newcommand{\coursewebpagelink}{\href{\coursewebpageurl}{course homepage}}
\newcommand{\slackurl}{https://QC\coursedept\coursenumber\semester\the\year.slack.com/}
\newcommand{\slacklink}{\href{\slackurl}{slack}}
\newcommand{\numtheoryhws}{7--10}
\newcommand{\extrahwzero}{}
\newcommand{\hwzerodue}{Wednesday, Feb 3 11:59PM}
\newcommand{\lastdatetimetohandinhomeworks}{May 18 at noon}
\newcommand{\midtermonedateandlocation}{Wednesday, March 3 on zoom during class time}
\newcommand{\midtermtwodateandlocation}{Wednesday, April 21 on zoom during class time}
\newcommand{\finaldateandlocation}{TBD but on zoom}

\input{../../syllabi/_header}

\section*{Course Overview}

MATH 341. Bayesian Modeling. 3 hr.; 3 cr. Prereq.: MATH 241. A review of frequentist methods followed by a survey of statistical modeling using the Bayesian framework: prior distribution design, including Jeffrey’s priors; likelihood models; posterior probabilities; hypothesis tests; Bayesian linear regression; Gibbs sampling; Metropolis-Hastings (basic Bayesian computing).  Emphasis on real-world applications, including those in finance and applied probability. The goal is to be fluent enough to understand how industry uses Bayesian modeling and computation by the end of the course.

Statistics has historically been taught from the frequentist perspective. Recently, the Bayesian perspective has become popular (1) due to their models' performance on previously intractable problems and the recent availability of inexpensive computational power and (2) it solves many philosophical quandaries in the Frequentist perspective. Further, many scientific journals are moving away from Frequentist p-values and confidence intervals in favor of the Bayesian analogues. It is imperative to learn this perspective as you will see these models in industry and this mode of thinking is becoming mainstream in science at large. \pagebreak


Here are the main topics to be covered:

\begin{itemize}
\itemsep -0.0em 
\item Data modeling with parametric families
\item Bayes Rule as it applies to parameters
\item prior distribution design 
\item Jeffrey's priors 
\item likelihood models and maximum likelihood estimators
\item posterior probabilities 
\item Bayesian inference: credibility intervals 
\item Bayesian inference: p-values for hypothesis tests
\item mixture priors 
\item mixture models
\item Newton-Raphson and Expectation-Maximization Algorithms
\item basic computing for Bayesian models including Gibbs sampling and Metropolis-Hastings Sampling 
\end{itemize}


\noindent The only prerequisite is Math 241 or equivalent. You should be familiar with the following:

\begin{itemize}
\itemsep -0.0em 
\item Basic Set Theory
\item Counting Methods | permutations and combinations
\item Basic Probability Theory | axioms, conditional probability, in/dependence
\item Modeling with Discrete Random Variables: Bernoulli, Hypergeometric, Binomial, Pois-
son, Geometric, Negative Binomial, Uniform Discrete and others
\item Expectation and Variance
\item Modeling with Continuous Random Variables: Exponential, Uniform and Normal
\item Frequentist Confidence Intervals and Hypothesis Testing for one-sample proportions
\end{itemize}

\noindent We will review the above \textit{throughout the semester} when needed and we will do so rapidly. \\

\textbf{This is not your typical mathematics course.} This course develops ideas and concepts for helping to make decisions based on randomness and we will do lots of modeling of real-world situations. The course does not dwell on theory nor details of computation but will make use of computation especially using the \texttt{R} statistical language.

\section*{Course Materials}

\paragraph{Textbook:} Introduction to Bayesian Statistics by William M. Bolstad First Edition. It can be purchased used on \href{http://www.amazon.com/gp/offer-listing/0471270202/ref=sr_1_2_twi_1_har_olp?ie=UTF8&qid=1433515305&sr=8-2&keywords=introduction+to+bayesian+statistics+bolstad}{Amazon}. This is \textit{recommended}. It is a way to get ``another take'' on the material. However, most of the material in the class comes from the lecture notes.

\paragraph{Popular Book:} We will also be reading the non-fiction novel \qu{The Theory that Would not Die: How Bayes' Rule Cracked the Enigma Code, Hunted Down Russian Submarines, and Emerged Triumphant from Two Centuries of Controversy} by Sharon Bertsch McGrayne which can also be purchased on \href{http://www.amazon.com/Theory-That-Would-Not-Die/dp/0300188226/ref=sr_1_1?ie=UTF8&qid=1454261896&sr=8-1&keywords=The+Theory+that+Would+not+Die}{Amazon}. This is \textit{required} --- you will have homework questions directly from this book.

\paragraph{Computer Software:} We will also be using \texttt{R} which is a free, open source statistical programming language and console. You can download it from: \url{http://cran.mirrors.hoobly.com/}. I do not expect you to do \textit{any} programming. I will be giving you \texttt{R} code to run and expect you to interpret the results based on concepts explained during the course.

\paragraph{Calculator:} You can use a TI-84, 85, 89 or any calculator which you wish. I strongly suggest you use \href{http://www.wolframalpha.com/}{Wolfram Alpha} and its smartphone app.

\input{../../syllabi/_the650section}

\input{../../syllabi/_announcements_on_slack}

\input{../../syllabi/_use_of_slack}

\input{../../syllabi/_standard_class_meetings}

I am \inred{canceling} Monday, May 17 (the last meeting) due to a Jewish holiday. This meeting would have been a review session for the final exam. We will decide when to have this session during finals week that fits the majority's schedule.

\input{../../syllabi/_zoom_policies}

\subsection*{Lecture Schedule}

Below is a tentative detailed enumeration of the lectures, topics with time estimates below:

%341 / 641: Introduction to Frequentist and Bayesian Statistical Inference, Coreq: 340
%343 / 643: Computational Statistical Inference and Experimentation, Coreq: 342

\begin{enumerate}
\item[Lec 1] [40min] Review of discrete and continuous random variables (rvs), support, probability mass functions (PMFs), cumulative distribution functions (CDFs), probability density functions (PDFs); joint mass functions (JMFs), joint density functions (JDFs), independence, iid multiplication rule; [20min] the Bernoulli rv, parameters, parameter space, degenerate rv; [5min] parametric models; [10min] statistical inference and its three goals

\item[Lec 2] [30min] the likelihood function, the log-likelihood function, example with iid Bernoulli, example with iid Geometric; [20min] the MLE and properties of the MLE; [10min] introduction to frequentist confidence intervals (CIs); [15min] intoduction to hypothesis testing and frequentist retainment regions

\item[Lec 3] [30min] Problems and limitations with frequentist CIs and testing, valid interpretation of frequentist CIs, the frequentist p-value; [35min] review of definition of conditional probability, Bayes Rule, Bayes Theorem; [10min] marginal and conditional PMFs

\item[Lec 4] [10min] Bayes rule for two rvs; [10min] anatomy of the Bayes identity: the likelihood, prior, prior predictive distribution and posterior distribution; [40min] example posterior calculation with discrete parameter space and principle of indifference; [15min] Bayesian point estimation with the maximum a posteriori (MAP) estimate, conditions for equivalence with the MLE

\item[Lec 5] [25min] Proof that Bayesian Inference is iterative in the data; [15min] uniform prior for the bernoulli iid model; [10min] Bayesian point estimation with the posterior median and posterior expectation; [20min] derivation of general beta posterior for the bernoulli iid model, intro to beta distribution, beta function, gamma function; [5min] point estimation with beta posterior


\item[Lec 6] [10min] all legal shapes of the beta distribution; [35min] the beta-binomial bayesian model; prior parameters (hyperparameters) and posterior parameters, point estimates; [10min] definition of conjugacy, beta-binomial conjugacy; [10min] pseudodata interpretation of the prior parameters; [10min] shrinkage estimators and the beta-binomial posterior expectation as a shrinkage estimator

\item[Lec 7] [15min] One-sided and two-sided credible regions (CRs); [5min] CR for beta-binomial model; [10min] high density regions; [20min] one-sided and two-sided Hypothesis testing and the decisions to reject or retain the null; [10min] decisions in the Bayesian framework for one-sided hypothesis testing, Bayesian p-values; [15min] beta-binomial examples


\item[Lec 8] [20min] two approaches for two-sided testing in the Bayesian framework; [40min] posterior predictive distribution formula, example for one future observation in the beta-binomial model; [15min] mixture and compound distributions

\item[Lec 9] [65min] the betabinomial distribution as an overdispersed binomial, example with birth data, proof of the general posterior predictive distribution for the beta-binomial model; [10min] Laplace and Haldane priors


\item[Lec 10] [25min] Informative priors for the beta-binomial model, example with baseball batting averages, shrinkage in informative priors, empiral Bayes estimation; [10min] definition of odds, reparameterization of the binomial with odds; [5min] PDF change of variables formula, proof that prior of indifference for binomial probability is not prior of indifference for odds; [15min] Jeffrey's prior specification concept; [10min] PDF/PMF decomposition into kernel and normalization constants; [10min] definition of Fisher information, computation of Fisher information for the binomial distribution

\item[Lec 11] [30min] Definition of Jeffrey's prior, derivation of Jeffrey's prior for the beta-binomial model, verification that it robust to reparameterizations of the binomial model's parameter; [10min] proof of Jeffrey's prior satisfies  Jeffrey's prior specification concept; [10min] derivation of Poisson model; [15min] derivation of the Poisson model's conjugate prior via kernel decomposition (the Gamma); [20min] Gamma shapes and properties

%-10min
\item[Lec 12] [15min] pseudodata interpretation of hyperparameters in the gamma-poisson model; [20min] derivation of the shrinkage point estimator for the gamma-poisson model; [10min] CRs for the gamma-poisson model; [20min] uninformative priors for the gamma-poisson model;

\item[Lec 13] [45min] derivation of the posterior predictive distribution being extended negative binomial in the gamma-poisson model; [15min] kernel decomposition of the normal PDF; [15min] Normal posterior under laplace prior


\item[Lec 14] [75min] derivation of the normal-normal conjugate model, pseudodata interpretation of the hyperparameters, Haldane prior, point estimation in the normal-normal model, Jeffrey's prior derivation, shrinkage estimator

\item[Lec 15] [40min] derivation of the normal posterior predictive distribution for the normal-normal model; [10min] derivation of the inversegamma distribution, properties of the inverse gamma distribution [35min] normal-inversegamma model, laplace prior, pseudodata interpretation of the hyperparameters, haldane prior

%-10
\item[Lec 16] [10min] point estimation in the normal-inversegamma model; [15min] Jeffrey's prior derivation for the normal-inversegamma model; [30min] derivation of the Student's T posterior predictive distribution for the normal-inversegamma model; [10min] shrinkage estimation in the normal-inversegamma model


\item[Lec 17] [75min] The two-dimensional  normal-inverse-gamma (NIG) distribution, its kernel, its use in bayesian inference for the conjugate NIG-NIG model

%+10
\item[Lec 18] [15min] Marginal mean T distribution in the NIG posterior; [15min] Marginal variance inverse-gamma distribution in the NIG posterior; [55min] derivation of the Student's T posterior predictive distribution in the NIG-NIG model

\item[Lec 19] [30min] Sampling from the NIG distribution; [45min] the kernel of the semiconjugate NIG model

\item[Lec 20] [35min] Grid sampling, distribution sampling via kernel grid sampling, disadvantages of grid sampling; [40min] systematic sweep Gibbs sampling, burning the chain, sampling from the semi-conjugate NIG model

\item[Lec 21] [20min] Autocorrelation grid sampling, thinning the chain; [30min] approximate inference with Gibbs samples; [25min] change point detection model

%+15min
\item[Lec 22] [55min] normal mixture model with data augmentation; [35min] Bayes Factors

%-15min
\item[Lec 23] [60min] Metropolis algorithm, Metropolis-Hastings algorithm, metropolis-within-Gibbs, transition kernels


\end{enumerate}

\input{../../syllabi/_lecture_upload}

\section*{Homework}

\input{../../syllabi/_theory_hws_text}

\input{../../syllabi/_theory_hws_submission_text}
\input{../../syllabi/_philosophy_hws}

\input{../../syllabi/_time_spent_hws}

\input{../../syllabi/_late_hw_policy}

\input{../../syllabi/_latex_hw_bonus_policy}

\input{../../syllabi/_hw_ec_policy}

\input{../../syllabi/_hw_0}

\section*{Examinations}

\input{../../syllabi/_examination_text}

\input{../../syllabi/_standard_exam_schedule}

\subsection*{Exam Policies and Materials}

\input{../../syllabi/_examination_policies}

\input{../../syllabi/_standard_cheat_sheet_policy}


\input{../../syllabi/_cheating_on_exams_and_missing_exams}
\input{../../syllabi/_special_services}

\input{../../syllabi/_class_participation}

\input{../../syllabi/_zoom_attendance}

\input{../../syllabi/_standard_grading_and_grading_policy}

\input{../../syllabi/_advanced_course_grade_distribution}

\input{../../syllabi/_grade_checking_on_gradesly}

\input{../../syllabi/_auditing_policy}

\end{document}
